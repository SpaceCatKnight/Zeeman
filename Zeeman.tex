\documentclass[a4paper,parskip,11pt, DIV12]{scrreprt}

\usepackage[ngerman]{babel} % Für Deutsch [english] zu [ngerman] ändern. 
\usepackage[utf8]{inputenc}
\usepackage[T1]{fontenc}
%\usepackage{blindtext}
\usepackage{graphicx}
\usepackage{subfigure}
\renewcommand{\familydefault}{\sfdefault}
\usepackage{helvet}
\usepackage{fancyhdr}
\usepackage{amsmath}
\usepackage{mdwlist} %Benötigt für Abstände in Aufzählungen zu löschen
\usepackage{here}
\usepackage{calc}
\usepackage{hhline}
\usepackage{marginnote}
\usepackage{chngcntr}
\usepackage{tabularx}
\usepackage{titlesec} % Textüberschriften anpassen

% \titleformat{Überschriftenklasse}[Absatzformatierung]{Textformatierung} {Nummerierung}{Abstand zwischen Nummerierung und Überschriftentext}{Code vor der Überschrift}[Code nach der Überschrift]

% \titlespacing{Überschriftenklasse}{Linker Einzug}{Platz oberhalb}{Platz unterhalb}[rechter Einzug]

\titleformat{\chapter}{\LARGE\bfseries}{\thechapter\quad}{0pt}{}
\titleformat{\section}{\Large\bfseries}{\thesection\quad}{0pt}{}
\titleformat{\subsection}{\large\bfseries}{\thesubsection\quad}{0pt}{}
\titleformat{\subsubsection}{\normalsize\bfseries}{\thesubsubsection\quad}{0pt}{}

\titlespacing{\chapter}{0pt}{-2em}{6pt}
\titlespacing{\section}{0pt}{6pt}{-0.2em}
\titlespacing{\subsection}{0pt}{5pt}{-0.4em}
\titlespacing{\subsubsection}{0pt}{-0.3em}{-1em}

%\usepackage[singlespacing]{setspace}
%\usepackage[onehalfspacing]{setspace}

\usepackage[
%includemp,				%marginalien in Textkörper einbeziehen
%includeall,
%showframe,				%zeigt rahmen zum debuggen		
marginparwidth=25mm, 	%breite der marginalien
marginparsep=5mm,		%abstand marginalien - text
reversemarginpar,		%marginalien links statt rechts
%left=50mm,				%abstand von Seitenraendern
%			top=25mm,				%
%			bottom=50mm,
]{geometry}		

%Bibliographie- Einstellungen
\usepackage[babel,german=quotes]{csquotes}
\usepackage[
backend=bibtex8, 
natbib=true,
style=numeric,
sorting=none
]{biblatex}
\bibliography{Quelle}
%Fertig Bibliographie- Einstellungen

\usepackage{hyperref}

\newenvironment{conditions}
{\par\vspace{\abovedisplayskip}\noindent\begin{tabular}{>{$}l<{$} @{${}={}$} l}}
	{\end{tabular}\par\vspace{\belowdisplayskip}}
	
\begin{document}
	
	\begin{titlepage}
		\begin{figure}[H]
			\hfill
			\subfigure{\includegraphics[scale=0.75]{uzh}}
		\end{figure}
		\vspace{1 cm}
		\textbf{\begin{huge}Praktikum zu Physik III
		\end{huge}}\\
		\noindent\rule{\textwidth}{1.1 pt} \\
		
		\begin{Large}\textbf{Zeeman-Effekt}
		\end{Large}\\ 
		\normalsize 
		\par
		\begingroup
		\leftskip 0 cm
		\rightskip\leftskip
		\textbf{Assistent:}\\ Josef Roos \\ \\
		\textbf{Studenten:}\\ Ruben Beynon, Manuel Sommerhalder, Stefan Hochrein \\ \\
		\textbf{Abgabetermin:}\\ 10.02.2017 \\ \\
		\par
		\endgroup
		\clearpage
		
		
		
	\end{titlepage}
	
%Start Layout
	\pagestyle{fancy}
	\fancyhead{} 
	\fancyhead[R]{\small \leftmark}
	\fancyhead[C]{\textbf{Praktikumsbericht Zeeman-Effekt} } 
	\fancyhead[L]{\includegraphics[height=2\baselineskip]{uzh}}
	
	\fancyfoot{}
	\fancyfoot[R]{\small \thepage}
	\fancyfoot[L]{}
	\fancyfoot[C]{}
	\renewcommand{\footrulewidth}{0.4pt} 
	
	\addtolength{\headheight}{2\baselineskip}
	\addtolength{\headheight}{0.6pt}
	
	
	\renewcommand{\headrulewidth}{0.6pt}
	\renewcommand{\footrulewidth}{0.4pt}
	\fancypagestyle{plain}{				% plain redefinieren, damit wirklich alle seiten im gleichen stil sind (ausser titlepage)
		\pagestyle{fancy}}
	
	\renewcommand{\chaptermark}[1]{ \markboth{#1}{} } %Das aktuelle Kapitel soll nicht Gross geschriben und Nummeriertwerden
	
	\counterwithout{figure}{chapter}
	\counterwithout{table}{chapter}
	%Ende Layout
	
	%\tableofcontents
	
		\chapter{Einführung}
	
	In diesem Praktikumsversuch wurde die Aufspaltung der Spektrallinien in einem externen Magnetfeld, auch Zeeman-Effekt genannt, untersucht. Der Zeeman-Effekt spaltet die ohne externes Magnetfeld entarteten Energiezustände in einzelne Energiezustände mit Energiedifferenz
	\begin{equation}
	\label{Energiedifferenz}
	\Delta E = g_j \mu_B \Delta m_j B 
	\end{equation}
auf. Ziel des Experiments ist es, den Landé-Faktor $g_j $ für bestimmte Übergänge zu bestimmen. 
	\section{Grundlagen}
	Das Experiment wurde anhand von zwei speziell ausgewählten Übergängen im Neon Atom durchgeführt, die beide im sichtbaren Bereich liegen. Beide untersuchten Übergänge führen von einem Zustand mit der Quantenzahl $ J = 0 $ zu einem Zustand mit $J = 1$. Da nur der $J = 1$ Zustand in drei Energiestufen aufgespaltet wird sind bei beiden Übergängen drei im Spektrum sehr nahe nebeneinander liegende Linien erkennbar. Im Verlauf des Experiments wurden mittels Interferenz-Spektrometer die Frequenzunterschiede $\Delta \nu$  dieser Linien ermittelt, und damit auch die Energieunterschiede $\Delta E = h \cdot \Delta \nu$. Durch zusätzliches Messen des Magnetfeldes kann so der Landé-Faktor bestimmt werden:
	\begin{equation}
	\label{Landé-Faktor}
	g_j = \frac{h \cdot \Delta \nu}{\mu_B \Delta m_j B}
	\end{equation}
	
	
	\chapter{Berechnungen}
	
	Wie in der Einfüehrung schon erwähnt sollten die Landé-Faktoren $g_j$ für zwei Spektrallinien bestimmt werden, die je aus drei nahe beieinander liegenden Linien bestehen. Da jeweils die innersten drei Maxima betrachtet wurden konnten auch drei $g_j$ pro Farbe berechnet werden. Die Formeln dazu stammen alle aus der Versuchsanleitung. 
	
	\section{Magnetfeld}
	
	Zur Bestimmung des Magnetfelds, in dem der Versuch stattgefunden hat, wurde mit einer Flipspule die Spannung $V$ gemmessen, die entsteht, wenn die Flipspule aus dem Magnetfeld gezogen wird. Da das Spannungsmessgerät die Spannung aufsummierte, und immer eine kleine Spannung gemessen wurde, driftete die Messung stark zu immer grösseren Spannungen. Deshalb wurde die Spannung je neun mal, mit der Flipspule aufrecht ins Magnetfeld gehalten (positive Spannung) und verkehrt ins Magnetfeld gehalten (negative Spannung) gemessen. Der Betrag der Messgrössen driftete so also für eine Messserie zu grösseren, für die andere zu kleineren Spannungen. Unter der Annahme das der Drift etwa konstant ist, wurde für die weitere Berechnung der Mittelwert aller Spannungen benutzt.   
	Das Magnetfeld berechnet sich dann durch die Formel
	\begin{equation}
	\label{B-Feld}
	B = \frac{V}{314.16 \cdot sa \cdot N} ,
	\end{equation}
	wobei $sa$ die Querschnittsfläche und $N$ die Anzahl Wicklungen der Spule sind. Da nicht bekannt ist, ob der Draht zur Querschnittsfläche gehört, wurde die Hälfte des Drahts zur Fläche gezählt. 
	
	\section{Frequenzunterschied}
	
	Zur Bestimmung des Frequenzunterschieds wurden die Positionen von drei, jeweils selbst in drei Linien aufgespalteten Maxima, gemessen. Die Positionen der Maxima wurden einmal links und einmal rechts der Lummerplatte, je vier mal, bestimmt. Der Durchschnitt dieser vier Messungen, wurde dann als endgültige Position benutzt. Damit konnte der Abstand $A$ zwischen den korrespondierenden Linien auf beiden Seiten der Platte berechnet werden. Zusäzlich wurde der Abstand $S$ von der Positionsbestimmung (Mikrometerschraube) zum Drehpunkt des Fernrohrs, auch durch mehrfaches Messen und anschliessendes Mitteln, bestimmt. 

Damit konnte der Winkel $\theta$ durch die Formel $\theta = \arctan(\frac{A}{2 S})$ berechnet werden und mit diesem dann weiter die Maximumsordnung  $M = \frac{2d}{\lambda} \cdot \sqrt{n^2-1+\sin^2(\theta)}$ , mit $d$ der Dicke der Lummerplatte, $\lambda$ der Wellenlänge der beobachteten Linie und $n$ dem Brechungsindex für ebendiese Wellenlänge. $\theta$ ist hier der Winkel für die Mittlere der drei nahe beieinander liegenden Linien. 

Nun konnte der Frequenzuaufspaltung $\Delta \nu$ berechnet werden:
	\begin{equation}
	\label{Frequenzunterschied}
	\Delta \nu = \frac{-c}{\lambda^2}\cdot\frac{\sin^2(\theta_m)-\sin^2(\theta_p)}{\frac{\lambda \cdot M^2}{d^2}-4\cdot n \cdot dn}
	\end{equation}
	
	\section{Landé-Faktor}
	
	Mit den nun berechneten Frequenzunterschieden und der bekannten Magnetfeldstärke konnten durch die schon in der Einleitung erwähnte Formel $\eqref{Landé-Faktor}$ die Landé-Faktoren $g_j$ bestimmt werde. Da pro Linienfarbe drei Maxima betrachtet wurden, wurden jeweils auch drei Werte für $g_j$ berechnet, die sich leicht unterscheiden. Als Endresultat wurde dann deren gewichteter Mittelwert mit der Formel  $\bar{x} = \frac{\sum\frac{x_i}{\sigma_i^2}}{\sum \frac{1}{\sigma_i^2}}$ berechnet. 
 
	 
	\chapter{Fehlerrechnung}
	Die ganze Fehlerrechnung wurde mittels Gauss'schem Fehlerfortpflanzungsgesetz 
	\begin{equation}
	\label{Fehlerfortpflanzungsgesetz}
	\Delta f(x,y,z,..) = \sqrt{(\frac{\partial f}{\partial x} \cdot \Delta x)^2+(\frac{\partial f}{\partial y} \cdot \Delta y)^2+(\frac{\partial f}{\partial z} \cdot \Delta z)^2+...}
	\end{equation}
	durchgeführt. Mögliche Korrelationen zwischen Messungen wurden also der Einfachheit halber nicht beachtet. Ausserdem wurden die auftretenden Naturkonstanten wie die Lichtgeschwindigkeit, die Planck'sche Konstante und das Bohr'sche Magneton als exakt angenommen und ihre Fehler nicht in die Fehlerrechnung mit einbezogen.  
	\section{Magnetfeld}
	Für den Fehler auf das Magnetfeld sind nur zwei Fehler verantwortlich. Einerseits der Fehler auf die Spannungsmessung und andererseits der Fehler auf die Querschnittsfläche der Flipspule. Der Umrechnungsfaktor und die Anzahl Wicklungen der Flipspule wurden als exakt angenommen. 
	
	Für den Fehler auf die Spannung wurden verschiedene Ansätze diskutiert. Das Problem bestand darin, dass das Spannungsmessgerät einen starken Drift hatte, die Spannung also ständig zunahm, auch wenn die Flipspule keine Spannung messen sollte. Mehrfaches Messen mit zuerst aufrechter, dann umgekehrter Flipspule, ergaben zwei Messdatensätze, einmal mit positiven, einmal mit negativen Spannungen. Durch den Drift wurden beide Datensätze in positive Richtung verschoben. Deshalb weist der Betrag aller Messdaten zwei Häufungspunkte auf. Die einfachste Möglichkeit, den Fehler zu besimmen, wäre, die Standardabweichung aller Messungen zu berechnen. Allerdings würde der Fehler so sehr gross. Die zweite Möglichkeit geht davon aus, dass der wahre Wert der Spannung sicher zwischen den beiden Häufungspunkten liegt. Als Fehler könnte dann die Hälfte des Abstandes zwischen den Häufungspunkten verstanden werden. Die letzte Möglichkeit wäre, die Messdaten so zurück zu verschieben, dass die Häufungspunkte wieder übereinander liegen, und dann die Standardabweichung zu berechnen. Der Fehler würde so allerdings sehr klein.
	
	Der Fehler auf die Querschnittsfläche der Flipspule rührt daher, dass nicht bekannt ist, ob die Fläche des Drahts zur Querschnittsfläche zählt. Deshalb wurde als Radius der Querschnittsfläche der Innenradius der Fläche plus die Hälfte des Drahltdurchmessers benutzt. Der Fehler au den Radius beträgt dann die Hälfte des Drahtdurchmessers. Mittels Gauss'scher Fehlerfortpflanzung wurde damit der Fehler auf die Querschnittsfläche berechnet. 
	
	Mit diesen Fehlern konnte nun wiederum mittels Fehlerfortpflanzungsgesetz der Fehler auf das Magnetfeld berechnet werden.
	 
	\section{Frequenzunterschied}
	Der Fehler des Frequenzunterschieds setzt sich aus vier Fehlern zusammen. Erstens dem Fehler auf den Abstand $A$, zweitens dem Fehler auf den Abstand $S$ , drittens dem Fehler des Brechungsindex $n$ der Lummerplatte und zuletzt dem Fehler auf die Steigung des Brechungsindex an der Stelle der entsprechenenden Wellenlängen. Die Fehler der anderen auftretenden Grössen, wie der Wellenlängen der Spektrallinien, und der Dicke der Lummerplatte wurden als Null, beziehungsweise so klein, dass sie keinen Einfluss haben, eingestuft. 
	
	Der Abstand $A$ entspricht der Differenz der beiden Positionen der entsprechenden Linien, links und rechts der Lummerplatte. Da für jede Position nur vier Messungen vorliegen, wurde die Standardabweichung der Messungen als Fehler auf die Position benutzt. Mittels Fehlerfortpflanzung wurde dann der Fehler von $A$ bestimmt.
	
	Auch für die Länge $S$, von der Mikrometerschraube zum Drehpunkt des Fernrohrs, liegen nur wenige Messungen vor. Deshalb wurde auch hier die Standardabweichung als Fehler benutzt.
	
	Um den Brechungsindex $n$ und seine Steigung zu bestimmen, wurde der Plot aus der Praktikumsanleitung benutzt. Der Fehler auf $n$ kommt durch die Skala und das ungenaue Ablesen zustande und wurde abgeschätzt.
Für die Steigung wurden von Hand neben einer besten, eine steilste und eine flachste Tangente an die Kurve gelegt und mit deren Hilfe der Fehler auf die Steigung abgeschätzt.

	Mittels Fehlerfortpflanzung wurden damit die Fehler auf die Winkel $\theta$, dann auf die Maximumsordnung $M$ und schliesslich auf den Frequenzunterschied $\Delta \nu$ bestimmt.
	
	\section{Landé-Faktor}
	
	Mit den nun bekannten Unsicherheiten des Magnetfelds und des Frequenzunterschieds, lässt sich für beide Linienfarben für jedes der drei beobachteten Maxima neben dem Landé-Faktor $g_j$ auch sein Fehler bestimmen. Schlussendlich wurde $g_j$ mit dem gewichteten Mittelwert ausgerechnet. Sein Fehler wurde deshalb mit der Fehlerformel für den gewichteten Mittelwert bestimmt. 
	\begin{equation}
	\label{Fehler gewichteter Mittelwert}
	\sigma_{\bar{x}} = \frac{1}{\sqrt{\sum \frac{1}{\sigma_{x_i}^2}}}
	\end{equation}
	
	\chapter{Resultate}
	
	\section{Gelb}
	Folgende Resultate für $g_j$ wurden berechnet:
	
\begin{tabular}{|c|c|c|c|}
\hline 
Linie  & $g_j$ & $\Delta g_j$ & Maximumsordnung \\ 
\hline 
M0 & 1.07 & 0.11 & 12492 \\ 
\hline 
M0+1 & 0.99 & 0.11 & 12493 \\ 
\hline 
M0+2 & 0.92 & 0.15 & 12494 \\ 
\hline 
Total & 1.01 & 0.07 & • \\ 
\hline 
\end{tabular} 

	\section{Blaugruen}
	Folgende Resultate für $g_j$ wurden berechnet:
	
\begin{tabular}{|c|c|c|c|}
\hline 
Linie  & $g_j$ & $\Delta g_j$ & Maximumsordnung \\ 
\hline 
M0 & 1.42 & 0.14 & 13577 \\ 
\hline 
M0+1 & 1.49 & 0.16 & 13578 \\ 
\hline 
M0+2 & 1.48 & 0.20 & 13579 \\ 
\hline 
Total & 1.46 & 0.09 & • \\ 
\hline 
\end{tabular} 

	\chapter{Diskussion}
	
	\chapter{Messdaten}
	In diesem Abschnitt befinden sich alle gemessenen Daten, mit denen die Auswertung durchgeführt wurde.
	
	\begin{table}[H]
	\begin{tabular}{|c|c|c||c|c|c||c|c|c|}
	\hline 
a	&	M0+2	&	i	&	a	&	M0+1	&	i	&	a	&	M0	&	i	\\
	\hline
	\hline
9.155	&	9.27	&	9.39	&	10.215	&	10.35	&	10.535	&	11.58	&	11.86	&	12.3	\\
	\hline
9.19	&	9.285	&	9.405	&	10.245	&	10.405	&	10.505	&	11.655	&	11.89	&	12.115	\\
	\hline
9.185	&	9.275	&	9.405	&	10.265	&	10.44	&	10.53	&	11.655	&	11.94	&	12.13	\\
	\hline
9.2	&	9.31	&	9.405	&	10.26	&	10.42	&	10.56	&	11.705	&	11.975	&	12.145	\\
	\hline
	\end{tabular}
	\caption{Gelbe Linien, links von der Lummerplatte}
	\end{table}
	
	\begin{table}[H]	
	\begin{tabular}{|c|c|c||c|c|c||c|c|c|}
	\hline 
i	&	M0	&	a	&	i	&	M0+1	&	a	&	i	&	M0+2	&	a	\\
	\hline
	\hline
17.8	&	18	&	18.245	&	19.275	&	19.425	&	19.61	&	20.43	&	20.535	&	20.58	\\
	\hline
17.785	&	17.975	&	18.21	&	19.34	&	19.455	&	19.61	&	20.46	&	20.53	&	20.555	\\
	\hline
17.755	&	17.97	&	18.2	&	19.305	&	19.42	&	19.595	&	20.425	&	20.515	&	20.555	\\
	\hline
17.79	&	17.99	&	18.185	&	19.255	&	19.415	&	19.585	&	20.41	&	20.515	&	20.555	\\
	\hline
	\end{tabular} 
	\caption{Gelbe Linien, rechts von der Lummerplatte}
	\end{table}
	
	\begin{table}[H]	
	\begin{tabular}{|c|c|c||c|c|c||c|c|c|}
	\hline 
a	&	M0+2	&	i	&	a	&	M0+1	&	i	&	a	&	M0	&	i	\\
	\hline
	\hline
9.225	&	9.335	&	9.555	&	10.275	&	10.46	&	10.635	&	11.64	&	11.82	&	12.1	\\
	\hline
9.285	&	9.375	&	9.58	&	10.205	&	10.45	&	10.705	&	11.525	&	11.835	&	12.135	\\
	\hline
9.25	&	9.46	&	9.59	&	10.24	&	10.47	&	10.695	&	11.56	&	11.945	&	12.215	\\
	\hline
9.24	&	9.45	&	9.66	&	10.3	&	10.565	&	10.745	&	11.6	&	11.97	&	12.23	\\
	\hline
	\end{tabular} 
	\caption{Blaugruene Linien, links von der Lummerplatte}
	\end{table}
	
	
	\begin{table}[H]
	\begin{tabular}{|c|c|c||c|c|c||c|c|c|}
	\hline 
i	&	M0	&	a	&	i	&	M0+1	&	a	&	i	&	M0+2	&	a	\\
	\hline
	\hline
17.805	&	18.03	&	18.33	&	19.21	&	19.38	&	19.59	&	20.315	&	20.44	&	20.58	\\
	\hline
17.685	&	17.968	&	18.27	&	19.165	&	19.345	&	19.57	&	20.23	&	20.395	&	20.555	\\
	\hline
17.675	&	17.86	&	18.265	&	19.14	&	19.315	&	19.525	&	20.2	&	20.36	&	20.555	\\
	\hline
17.595	&	17.885	&	18.225	&	19.085	&	19.28	&	19.53	&	20.17	&	20.32	&	20.555	\\
	\hline
	\end{tabular} 
	\caption{Blaugruene Linien, rechts von der Lummerplatte}
	\end{table}
	
	\begin{table}[H]
	\begin{tabular}{|c|c|c|c|c|c|c|}
	\hline 
Messung Nr.	&	1	&	2	&	3	&	4	&	5	&	6	\\
\hline
S [mm]	&	230.5	&	231	&	229	&	230.5	&	231.5	&	230	\\
	\hline
	\end{tabular} 
	\caption{Messung der Länge S}
	\end{table}
	
	\begin{table}[H]
	\begin{tabular}{|c|c|c|c|c|c|c|c|c|c|c|}
	\hline 
Messung Nr.	&	1	&	2	&	3	&	4	&	5	&	6	&	7	&	8	&	9	&	10	\\
\hline
U [V] vorwärts	&	4.53	&	4.48	&	4.46	&	4.35	&	4.52	&	4.53	&	4.47	&	4.53	&	4.51	&	4.48	\\
\hline
U [V] rückwärts	&	-4.28	&	-4.25	&	-4.21	&	-4.28	&	-4.32	&	-4.25	&	-4.23	&	-4.26	&	-4.26	&	-4.24	\\
	\hline
	\end{tabular} 
	\caption{Messung der Spannung V}
	\end{table}
\end{document}
