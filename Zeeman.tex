\documentclass[a4paper,parskip,11pt, DIV12]{scrreprt}

\usepackage[ngerman]{babel} % Für Deutsch [english] zu [ngerman] ändern. 
\usepackage[utf8]{inputenc}
\usepackage[T1]{fontenc}
%\usepackage{blindtext}
\usepackage{graphicx}
\usepackage{subfigure}
\renewcommand{\familydefault}{\sfdefault}
\usepackage{helvet}
\usepackage{fancyhdr}
\usepackage{amsmath}
\usepackage{mdwlist} %Benötigt für Abstände in Aufzählungen zu löschen
\usepackage{here}
\usepackage{calc}
\usepackage{hhline}
\usepackage{marginnote}
\usepackage{chngcntr}
\usepackage{tabularx}
\usepackage{titlesec} % Textüberschriften anpassen

% \titleformat{Überschriftenklasse}[Absatzformatierung]{Textformatierung} {Nummerierung}{Abstand zwischen Nummerierung und Überschriftentext}{Code vor der Überschrift}[Code nach der Überschrift]

% \titlespacing{Überschriftenklasse}{Linker Einzug}{Platz oberhalb}{Platz unterhalb}[rechter Einzug]

\titleformat{\chapter}{\LARGE\bfseries}{\thechapter\quad}{0pt}{}
\titleformat{\section}{\Large\bfseries}{\thesection\quad}{0pt}{}
\titleformat{\subsection}{\large\bfseries}{\thesubsection\quad}{0pt}{}
\titleformat{\subsubsection}{\normalsize\bfseries}{\thesubsubsection\quad}{0pt}{}

\titlespacing{\chapter}{0pt}{-2em}{6pt}
\titlespacing{\section}{0pt}{6pt}{-0.2em}
\titlespacing{\subsection}{0pt}{5pt}{-0.4em}
\titlespacing{\subsubsection}{0pt}{-0.3em}{-1em}

%\usepackage[singlespacing]{setspace}
%\usepackage[onehalfspacing]{setspace}

\usepackage[
%includemp,				%marginalien in Textkörper einbeziehen
%includeall,
%showframe,				%zeigt rahmen zum debuggen		
marginparwidth=25mm, 	%breite der marginalien
marginparsep=5mm,		%abstand marginalien - text
reversemarginpar,		%marginalien links statt rechts
%left=50mm,				%abstand von Seitenraendern
%			top=25mm,				%
%			bottom=50mm,
]{geometry}		

%Bibliographie- Einstellungen
\usepackage[babel,german=quotes]{csquotes}
\usepackage[
backend=bibtex8, 
natbib=true,
style=numeric,
sorting=none
]{biblatex}
\bibliography{Quelle}
%Fertig Bibliographie- Einstellungen

\usepackage{hyperref}

\newenvironment{conditions}
{\par\vspace{\abovedisplayskip}\noindent\begin{tabular}{>{$}l<{$} @{${}={}$} l}}
	{\end{tabular}\par\vspace{\belowdisplayskip}}
	

\begin{document}
	
	\begin{titlepage}
		\begin{figure}[H]
			\hfill
			\subfigure{\includegraphics[scale=0.75]{uzh}}
		\end{figure}
		\vspace{1 cm}
		\textbf{\begin{huge}Praktikum zu Physik III
		\end{huge}}\\
		\noindent\rule{\textwidth}{1.1 pt} \\
		
		\begin{Large}\textbf{Zeeman-Effekt}
		\end{Large}\\ 
		\normalsize 
		\par
		\begingroup
		\leftskip 0 cm
		\rightskip\leftskip
		\textbf{Assistent:}\\ Josef Roos \\ \\
		\textbf{Studenten:}\\ Ruben Beynon, Manuel Sommerhalder, Stefan Hochrein \\ \\
		\textbf{Abgabetermin:}\\ 10.02.2017 \\ \\
		\par
		\endgroup
		\clearpage
		
		
		
	\end{titlepage}
	
%Start Layout
	\pagestyle{fancy}
	\fancyhead{} 
	\fancyhead[R]{\small \leftmark}
	\fancyhead[C]{\textbf{Praktikumsbericht Zeeman-Effekt} } 
	\fancyhead[L]{\includegraphics[height=2\baselineskip]{uzh}}
	
	\fancyfoot{}
	\fancyfoot[R]{\small \thepage}
	\fancyfoot[L]{}
	\fancyfoot[C]{}
	\renewcommand{\footrulewidth}{0.4pt} 
	
	\addtolength{\headheight}{2\baselineskip}
	\addtolength{\headheight}{0.6pt}
	
	
	\renewcommand{\headrulewidth}{0.6pt}
	\renewcommand{\footrulewidth}{0.4pt}
	\fancypagestyle{plain}{				% plain redefinieren, damit wirklich alle seiten im gleichen stil sind (ausser titlepage)
		\pagestyle{fancy}}
	
	\renewcommand{\chaptermark}[1]{ \markboth{#1}{} } %Das aktuelle Kapitel soll nicht Gross geschriben und Nummeriertwerden
	
	\counterwithout{figure}{chapter}
	\counterwithout{table}{chapter}
	%Ende Layout
	
	%\tableofcontents
	
		\chapter{Einführung}
	
	In diesem Praktikumsversuch wurde die Aufspaltung der Spektrallinien in einem externen Magnetfeld, auch Zeeman-Effekt genannt, untersucht. Der Zeeman-Effekt spaltet die ohne externes Magnetfeld entarteten Energiezustände in einzelne Energiezustände mit Energiedifferenz
	\begin{equation}
	\label{Energiedifferenz}
	\Delta E = g_j \mu_B \Delta m_j B 
	\end{equation}
auf. Ziel des Experiments ist es, den Landé-Faktor $g_j $ für bestimmte Übergänge zu bestimmen. 

	\section{Grundlagen}
	Das Experiment wurde anhand von zwei speziell ausgewählten Übergängen im Neon Atom durchgeführt, die beide im sichtbaren Bereich liegen. Beide untersuchten Übergänge führen von einem Zustand mit der Quantenzahl $ J = 0 $ zu einem Zustand mit $J = 1$. Da nur der $J = 1$ Zustand in drei Energiestufen aufgespaltet wird sind bei beiden Übergängen drei im Spektrum sehr nahe nebeneinander liegende Linien erkennbar. Im Verlauf des Experiments wurden mittels Interferenz-Spektrometer die Frequenzunterschiede $\Delta \nu$  dieser Linien ermittelt, und damit auch die Energieunterschiede $\Delta E = h \cdot \Delta \nu$. Durch zusätzliches Messen des Magnetfeldes kann so der Landé-Faktor bestimmt werden:
	\begin{equation}
	\label{Landé-Faktor}
	g_j = \frac{h \cdot \Delta \nu}{\mu_B \Delta m_j B}
	\end{equation}
		\section{Berechnungen}
	\section{Fehlerrechnung}
	\section{Messdaten}
	In diesem Abschnitt befinden sich alle gemessenen Daten, mit denen die Auswertung ausgefuehrt wurde.
	
	\begin{table}
	\begin{tabular}{|c|c|c||c|c|c||c|c|c|}
	\hline 
a	&	M0+2	&	i	&	a	&	M0+1	&	i	&	a	&	M0	&	i	\\
	\hline
	\hline
9.155	&	9.27	&	9.39	&	10.215	&	10.35	&	10.535	&	11.58	&	11.86	&	12.3	\\
	\hline
9.19	&	9.285	&	9.405	&	10.245	&	10.405	&	10.505	&	11.655	&	11.89	&	12.115	\\
	\hline
9.185	&	9.275	&	9.405	&	10.265	&	10.44	&	10.53	&	11.655	&	11.94	&	12.13	\\
	\hline
9.2	&	9.31	&	9.405	&	10.26	&	10.42	&	10.56	&	11.705	&	11.975	&	12.145	\\
	\hline
	\end{tabular}
	\caption{Gelbe Linien, links von der Lummerplatte}
	\end{table}
	
	\begin{table}	
	\begin{tabular}{|c|c|c||c|c|c||c|c|c|}
	\hline 
i	&	M0	&	a	&	i	&	M0+1	&	a	&	i	&	M0+2	&	a	\\
	\hline
	\hline
17.8	&	18	&	18.245	&	19.275	&	19.425	&	19.61	&	20.43	&	20.535	&	20.58	\\
	\hline
17.785	&	17.975	&	18.21	&	19.34	&	19.455	&	19.61	&	20.46	&	20.53	&	20.555	\\
	\hline
17.755	&	17.97	&	18.2	&	19.305	&	19.42	&	19.595	&	20.425	&	20.515	&	20.555	\\
	\hline
17.79	&	17.99	&	18.185	&	19.255	&	19.415	&	19.585	&	20.41	&	20.515	&	20.555	\\
	\hline
	\end{tabular} 
	\caption{Gelbe Linien, rechts von der Lummerplatte}
	\end{table}
	

	\begin{table}	
	\begin{tabular}{|c|c|c||c|c|c||c|c|c|}
	\hline 
a	&	M0+2	&	i	&	a	&	M0+1	&	i	&	a	&	M0	&	i	\\
	\hline
	\hline
9.225	&	9.335	&	9.555	&	10.275	&	10.46	&	10.635	&	11.64	&	11.82	&	12.1	\\
	\hline
9.285	&	9.375	&	9.58	&	10.205	&	10.45	&	10.705	&	11.525	&	11.835	&	12.135	\\
	\hline
9.25	&	9.46	&	9.59	&	10.24	&	10.47	&	10.695	&	11.56	&	11.945	&	12.215	\\
	\hline
9.24	&	9.45	&	9.66	&	10.3	&	10.565	&	10.745	&	11.6	&	11.97	&	12.23	\\
	\hline
	\end{tabular} 
	\caption{Blaugruene Linien, links von der Lummerplatte}
	\end{table}
	
	
	\begin{table}
	\begin{tabular}{|c|c|c||c|c|c||c|c|c|}
	\hline 
i	&	M0	&	a	&	i	&	M0+1	&	a	&	i	&	M0+2	&	a	\\
	\hline
	\hline
17.805	&	18.03	&	18.33	&	19.21	&	19.38	&	19.59	&	20.315	&	20.44	&	20.58	\\
	\hline
17.685	&	17.968	&	18.27	&	19.165	&	19.345	&	19.57	&	20.23	&	20.395	&	20.555	\\
	\hline
17.675	&	17.86	&	18.265	&	19.14	&	19.315	&	19.525	&	20.2	&	20.36	&	20.555	\\
	\hline
17.595	&	17.885	&	18.225	&	19.085	&	19.28	&	19.53	&	20.17	&	20.32	&	20.555	\\
	\hline
	\end{tabular} 
	\caption{Blaugruene Linien, rechts von der Lummerplatte}
	\end{table}
	
\end{document}
